\documentclass[fleqn,11pt,a4paper,dvipdfmx]{jsarticle}
%
\usepackage{amsmath,amssymb}
\usepackage{bm}
\usepackage[dvipdfmx]{graphicx}
\usepackage{bmpsize}  % ← バウンディングボックス用
\usepackage{ascmac}
\usepackage{multicol} 
\usepackage{paracol}
\usepackage{tikz}
\usepackage{caption}
\usetikzlibrary{calc}
%
% \setlength{\textwidth}{\fullwidth}
% \setlength{\textheight}{39\baselineskip}
% \addtolength{\textheight}{\topskip}
% \setlength{\voffset}{-0.5in}
% \setlength{\headsep}{0.3in}
% \setlength{\mathindent}{0pt}  % 数式の左端インデントを0に
\vspace*{-\baselineskip} % ← 不要な最初の空白を詰める
\usepackage[
  left=2cm,    % 左だけ広め
  right=2cm,   % 右は狭め
  top=2cm,     % 上も狭め
  bottom=1.5cm   % 下も狭め
]{geometry}
%
\newcommand{\divergence}{\mathrm{div}\,}  %ダイバージェンス
\newcommand{\grad}{\mathrm{grad}\,}  %グラディエント
\newcommand{\rot}{\mathrm{rot}\,}  %ローテーション
%
% \pagestyle{myheadings}
\begin{document}
%
%
\section{正弦波電源電圧を用いた非線形回路}

家庭用電力定格における非線形負荷の使用は、1970年代に大幅に増加した。
テレビ受信機やラジオ受信機、調光器、多段変速電動機器などの機器には、通常、制御整流素子または非制御整流素子が組み込まれている。
テレビ受信機には入力整流器が備わっており、電源電流には直流成分が含まる。
図5.1に示すモノクロテレビの典型的な電源電流オシログラムは、波形の半波特性を示している。

図5.1 モノクロテレビ受信機の電源電流波形

これには、9次高調波までの大きな高調波電流と直流電流項(図5.2)が含まれてる。
英国製の標準的な21インチカラー受信機では、図5.3の波形が測定され、50Hzのスパイクのピーク値は7Aである。
最近のカラー受信機では、電源ラインから直流電流を除去するために、電源に全波整流器の入力段が装備されている。
典型的な電源電流波形を図5.4のオシログラムに示す。
例えば1976年には、英国には1,800万台のテレビ受信機があり、そのうち約半数がカラー受信機であった。

電気アーク炉やサイリスタ制御モーターなどの産業用負荷は、配電系統に大きな非線形インピーダンスを生じさせる。
例えば、アーク炉の負荷サイクルの最初の部分(3~8時間)は溶融期間と呼ばれ、固体装入物が溶融する。
この期間は、アークの不安定性とスクラップ金属の移動によって引き起こされる激しい電流変動を特徴である。
これらの電流変動は不規則で正確に予測することは不可能であり、供給システムのインピーダンスにおける電流の変動により、他の電力消費機器への供給電圧の低下を引き起こす可能性がある。
しかしながら、非線形回路におけるエネルギーの流れと電力分配を研究するためには、電源から非正弦波の電流が流れているにもかかわらず、供給電圧がほぼ正弦波のままであるアプリケーション群を区別することが有用である。
このようなシステムは、エネルギーの観点から興味深い特性を持つことが分かっており、これらの特性は力率改善の可能性のある方法について有用な知見をもたらす。

図5.2 図5.1の供給電流波形の高調波成分


\newpage
\begin{table}[ht]
  \begin{tabular}{l|ll}
  単語                & 品詞      & 日本語訳          \\ \hline
  distribution        & (不可算)名詞 & 配分,配給      \\
  power distribution  &         & 配電                \\ \hline
  component           & 名詞      & 構成要素,成分     \\
  reactive components &         & 無効成分            \\ \hline
  shunt               & 他動詞  & 入れ替える,変える   \\
                      & 名詞    & 分岐器              \\
  shunted             & 過去分詞& 分路,短絡           \\ 
  shunt circuit       &         & 分岐(並列)回路,分路 \\ \hline
  
  
  \end{tabular}
\end{table}

%
%
\end{document}