\documentclass[fleqn,11pt,a4paper,dvipdfmx]{jsarticle}
%
\usepackage{amsmath,amssymb}
\usepackage{bm}
\usepackage[dvipdfmx]{graphicx}
\usepackage{bmpsize}  % ← バウンディングボックス用
\usepackage{ascmac}
\usepackage{multicol} 
\usepackage{paracol}
\usepackage{tikz}
\usepackage{caption}
\usetikzlibrary{calc}
%
% \setlength{\textwidth}{\fullwidth}
% \setlength{\textheight}{39\baselineskip}
% \addtolength{\textheight}{\topskip}
% \setlength{\voffset}{-0.5in}
% \setlength{\headsep}{0.3in}
% \setlength{\mathindent}{0pt}  % 数式の左端インデントを0に
\vspace*{-\baselineskip} % ← 不要な最初の空白を詰める
\usepackage[
  left=2cm,    % 左だけ広め
  right=2cm,   % 右は狭め
  top=2cm,     % 上も狭め
  bottom=1.5cm   % 下も狭め
]{geometry}
%
\newcommand{\divergence}{\mathrm{div}\,}  %ダイバージェンス
\newcommand{\grad}{\mathrm{grad}\,}  %グラディエント
\newcommand{\rot}{\mathrm{rot}\,}  %ローテーション
%
% \pagestyle{myheadings}
\begin{document}
%
%
\section{正弦波電源電圧を用いた非線形回路}

家庭用電力定格における非線形負荷の使用は、1970年代に大幅に増加した。
テレビ受信機やラジオ受信機、調光器、多段変速電動機器などの機器には、通常、制御整流素子または非制御整流素子が組み込まれている。
テレビ受信機には入力整流器が備わっており、電源電流には直流成分が含まる。
図5.1に示すモノクロテレビの典型的な電源電流オシログラムは、波形の半波特性を示している。

図5.1 モノクロテレビ受信機の電源電流波形

これには、9次高調波までの大きな高調波電流と直流電流項(図5.2)が含まれてる。
英国製の標準的な21インチカラー受信機では、図5.3の波形が測定され、50Hzのスパイクのピーク値は7Aである。
最近のカラー受信機では、電源ラインから直流電流を除去するために、電源に全波整流器の入力段が装備されている。
典型的な電源電流波形を図5.4のオシログラムに示す。
例えば1976年には、英国には1,800万台のテレビ受信機があり、そのうち約半数がカラー受信機であった。

電気アーク炉やサイリスタ制御モーターなどの産業用負荷は、配電系統に大きな非線形インピーダンスを生じさせる。
例えば、アーク炉の負荷サイクルの最初の部分(3~8時間)は溶融期間と呼ばれ、固体装入物が溶融する。
この期間は、アークの不安定性とスクラップ金属の移動によって引き起こされる激しい電流変動を特徴である。
これらの電流変動は不規則で正確に予測することは不可能であり、供給システムのインピーダンスにおける電流の変動により、他の電力消費機器への供給電圧の低下を引き起こす可能性がある。
しかしながら、非線形回路におけるエネルギーの流れと電力分配を研究するためには、電源から非正弦波の電流が流れているにもかかわらず、供給電圧がほぼ正弦波のままであるアプリケーション群を区別することが有用である。
このようなシステムは、エネルギーの観点から興味深い特性を持つことが分かっており、これらの特性は力率改善の可能性のある方法について有用な知見をもたらす。

図5.2 図5.1の供給電流波形の高調波成分


\newpage
% Please add the following required packages to your document preamble:
% \usepackage[table,xcdraw]{xcolor}
% Beamer presentation requires \usepackage{colortbl} instead of \usepackage[table,xcdraw]{xcolor}
\documentclass[fleqn,9pt,a4paper,dvipdfmx]{jsarticle}
%
\usepackage{amsmath,amssymb}
\usepackage{bm}
\usepackage[dvipdfmx]{graphicx}
\usepackage{bmpsize}  % ← バウンディングボックス用
\usepackage{ascmac}
\usepackage{multicol} 
\usepackage{paracol}
\usepackage{tikz}
\usepackage{caption}
\usepackage[table,xcdraw]{xcolor}
\vspace*{-\baselineskip} % ← 不要な最初の空白を詰める
\usepackage[
  left=2cm,    % 左だけ広め
  right=2cm,   % 右は狭め
  top=2cm,     % 上も狭め
  bottom=1.5cm   % 下も狭め
]{geometry}
%
\newcommand{\divergence}{\mathrm{div}\,}  %ダイバージェンス
\newcommand{\grad}{\mathrm{grad}\,}  %グラディエント
\newcommand{\rot}{\mathrm{rot}\,}  %ローテーション
%
% \pagestyle{myheadings}
\begin{document}
% Please add the following required packages to your document preamble:
% \usepackage[table,xcdraw]{xcolor}
% Beamer presentation requires \usepackage{colortbl} instead of \usepackage[table,xcdraw]{xcolor}
  \begin{enumerate}
    \item Instantaneous relationships for series-parallel R-L-C load
  
  \begin{table}[h]
    \begin{tabular}{l|ll}
    単語            & 品詞    & 日本語訳                 \\ \hline
    Instantaneous & 形容詞   & 瞬間の,即時の              \\ \hline
    associated    & 他動詞   & 関係づける,関連づけて考える,連想する. \\ \hline
                  & 自動詞   & 〔…と〕提携する,連携する,連合する   \\
                  & 可算名詞  & 仲間; 同僚.              \\
                  & 形容詞   & 連合した,仲間の.            \\ \hline
    particular    & 形容詞   & 特にこの, 特有の            \\ \hline
    dissipation   & 不可算名詞 & 消散,消失,浪費             \\ \hline
    invariant     & 名詞    & 不変; 不変関係; 不変量       \\ \hline
    distribution        & (不可算)名詞 & 配分,配給      \\
    power distribution  &         & 配電                \\ \hline
    component           & 名詞      & 構成要素,成分     \\
    reactive components &         & 無効成分            \\ \hline
    shunt               & 他動詞  & 入れ替える,変える   \\
                        & 名詞    & 分岐器              \\
    shunted             & 過去分詞& 分路,短絡           \\ 
    shunt circuit       &         & 分岐(並列)回路,分路 \\ \hline
    \end{tabular}
    \end{table}
    \newpage

    \item Nonlinear circuits with sinusoidal supply voltage
\begin{table}[h]
  \begin{tabular}{l|ll}
  単語                         & 品詞                             & 意味                             \\ \hline
  nonlinear                  & 形容詞                            & 非線形の                           \\ \hline
  dimmer                     & 可算名詞                           & 薄暗くするもの,(証明の)調光器,制光装置          \\
  light dimmers              &                                & 調光器                            \\ \hline
  multi-speed electric motor &                                & 多段速度電動機                        \\ \hline
  appliances                 & 可算名詞                           & (特に,家庭用の)器具,装置,設備,電気製品         \\ \hline
  incorporate                & 他動詞                            & (…に)組み入れる,(…を)(…と)合体させる,合同させる \\ \hline
  incorporate controlled     &                                & 制御されたものを組み込む                   \\ \hline
  rectifier                  & 可算名詞                           & 整流器.                           \\ \hline
  oscillogtam                & 名詞                             & オシログラフによって生み出される記録             \\ \hline
  monochrome                 & 形容詞                            & 単色の,白黒の                        \\
                             & 可算名詞                           & 単色画; 黒白写真,モノクロ(写真)            \\
                             & 不可算名詞                          & 単色画{[}写真{]}法                  \\ \hline
  waveform                   & 名詞                             & 波形                             \\ \hline
  signficant (harmonic)      & 形容詞                            & 重要な,意義深い                      \\ \hline
  colour                     & 不可算名詞                          & 色                              \\ \hline
                             & 他動詞                            & 〈…に〉彩色する,色を塗る; 染める            \\
                             & 形容詞                            & 色の,色彩の                        \\ \hline
  measured                   & 形容詞                            & 熟慮した,慎重な                      \\ \hline
  amp                        &                                 & アンペア                             \\\hline
  eliminate                  & 他動詞                            & 〈…を〉除く,除去する                   \\ \hline
  roughly                    & 副詞                             & ざっと,概略的に                       \\ \hline
  industrial(loads)          & 形容詞                            & 産業(上)の,工業(上)の,工業用の            \\ \hline
  furnace                    & 可算名詞                           & 炉,かまど.暖房炉                     \\ \hline
  thyristor                  &                                & 電流を制御することができる半導体素子             \\ \hline
  melt-down                  & 他動詞+ 副詞 & 〈地金などを〉溶かす,   鋳つぶす            \\
                             & 自動詞+ 副詞                        & 溶ける, 溶解する                     \\ \hline
  solid                      & 形容詞                            & 固体の,固形体の,学問など〉基礎のしっかりした; 堅実な  \\ \hline
  characterised              & 動詞                             & 特徴づける                          \\ \hline
  severe                     & 形容詞                            & 厳しい,厳格な                        \\ \hline
  fluctuation                & 名詞                             & 高下,変動                        \\ \hline
  arc                        & 可算名詞                           & 弧,円弧; 【電気】 アーク,電弧.             \\ \hline
  instability                & 不可算名詞                          & 不安定(性),                        \\ \hline
  scrap                      & 名詞                             & 切れ端{[}否定文で{]} 少し,鉄くず,スクラップ     \\
                             & 形容詞                            & 小片の,断片からなる                    \\
                             & 他動詞                            & くず鉄にする                        \\ \hline
  precisely                  & 副詞                             & 正確に,精密に                        \\ \hline
  predict                    & 他動詞                            & (…を)予言する,予報する                  \\ \hline
  dips                       & 他動詞                            & ちょっと浸す                        \\
                             & 自動詞                            & 沈む,下がる                        \\ \hline
  distinguish                & 他動詞                            & はっきり区別する,見分ける,聞き分ける \\
  \end{tabular}
  \end{table}
\end{enumerate}
  \end{document}

%
%
\end{document}