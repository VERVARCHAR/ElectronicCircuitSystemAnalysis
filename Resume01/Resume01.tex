\documentclass[fleqn,11pt,a4paper,dvipdfmx]{jsarticle}
%
\usepackage{amsmath,amssymb}
\usepackage{bm}
\usepackage[dvipdfmx]{graphicx}
\usepackage{bmpsize}  % ← バウンディングボックス用
\usepackage{ascmac}
\usepackage{multicol} 
\usepackage{paracol}
\usepackage{tikz}
\usepackage{caption}
\usetikzlibrary{calc}
%
% \setlength{\textwidth}{\fullwidth}
% \setlength{\textheight}{39\baselineskip}
% \addtolength{\textheight}{\topskip}
% \setlength{\voffset}{-0.5in}
% \setlength{\headsep}{0.3in}
% \setlength{\mathindent}{0pt}  % 数式の左端インデントを0に
\vspace*{-\baselineskip} % ← 不要な最初の空白を詰める
\usepackage[
  left=2cm,    % 左だけ広め
  right=2cm,   % 右は狭め
  top=2cm,     % 上も狭め
  bottom=1.5cm   % 下も狭め
]{geometry}
%
\newcommand{\divergence}{\mathrm{div}\,}  %ダイバージェンス
\newcommand{\grad}{\mathrm{grad}\,}  %グラディエント
\newcommand{\rot}{\mathrm{rot}\,}  %ローテーション
%
% \pagestyle{myheadings}
\begin{document}
%
%
\section{直並列R-L-C回路の瞬時関係}


図3.5は、直列R-C分岐によって分流された直列R-L分岐を持つR-L-C回路である。

この形式の回路が特に興味深いのは、Hallenが報告しているように、抵抗R1,R2のエネルギー散逸の瞬間的な割合を時間不変にすることができるからである。

図3.5の回路方程式は、理想的な正弦波電圧供給で、


\newpage
\begin{table}[ht]
  \begin{tabular}{l|ll}
  単語                & 品詞      & 日本語訳          \\ \hline
  distribution        & (不可算)名詞 & 配分,配給      \\
  power distribution  &         & 配電                \\ \hline
  component           & 名詞      & 構成要素,成分     \\
  reactive components &         & 無効成分            \\ \hline
  shunt               & 他動詞  & 入れ替える,変える   \\
                      & 名詞    & 分岐器              \\
  shunted             & 過去分詞& 分路,短絡           \\ 
  shunt circuit       &         & 分岐(並列)回路,分路 \\ \hline
  
  
  \end{tabular}
\end{table}

%
%
\end{document}