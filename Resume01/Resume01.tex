\documentclass[12pt, oneside]{book}

% --- 日本語とフォントの設定(必要なら) ---
\usepackage[utf8]{inputenc}
\usepackage[T1]{fontenc}
\usepackage{lmodern}

% --- ページと余白の設定 ---
\usepackage[a4paper, margin=2.5cm]{geometry}

% --- 図や色など ---
\usepackage{graphicx}
\usepackage{xcolor}
\usepackage{caption}
\usepackage{subcaption}

% --- 数式用 ---
\usepackage{amsmath, amssymb}


% --- 目次で日本語が文字化けしないように ---
\usepackage{hyperref}
\hypersetup{
    colorlinks=true,
    linkcolor=blue,
    urlcolor=cyan
}

\title{電子回路入門}
\author{山田 太郎}
\date{\today}

\begin{document}

% --- 表紙 ---
\maketitle
\tableofcontents
\newpage

% --- 本文(例) ---
\chapter{直流回路の基礎}
\section{オームの法則}
電流 \( I \) は電圧 \( V \) を抵抗 \( R \) で割ったものとして定義されます:

\[
I = \frac{V}{R}
\]

\section{回路図}

\chapter{交流回路の基礎}
\section{正弦波電流と電圧}
交流では時間 \( t \) に対して電圧は次のように表されます:

\[
V(t) = V_0 \sin(\omega t + \phi)
\]

\end{document}
