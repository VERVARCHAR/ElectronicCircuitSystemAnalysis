\documentclass[fleqn,11pt,a4paper,dvipdfmx]{jsarticle}
%
\usepackage{amsmath,amssymb}
\usepackage{bm}
\usepackage[dvipdfmx]{graphicx}
\usepackage{bmpsize}  % ← バウンディングボックス用
\usepackage{ascmac}
\usepackage{multicol} 
\usepackage{paracol}
\usepackage{tikz}
\usepackage{caption}
\usepackage{amsmath}
\usepackage{mathtools}
\captionsetup[table]{justification=centering}
\captionsetup[figure]{justification=centering}
\usetikzlibrary{calc}
\renewcommand{\thefigure}{\thesection.\arabic{figure}}
% \setlength{\textwidth}{\fullwidth}
% \setlength{\textheight}{39\baselineskip}
% \addtolength{\textheight}{\topskip}
% \setlength{\voffset}{-0.5in}
% \setlength{\headsep}{0.3in}
% \setlength{\mathindent}{0pt}  % 数式の左端インデントを0に
\vspace*{-\baselineskip} % ← 不要な最初の空白を詰める
\usepackage[
  left=2cm,    % 左だけ広め
  right=2cm,   % 右は狭め
  top=2cm,     % 上も狭め
  bottom=1.5cm   % 下も狭め
]{geometry}
%
\newcommand{\divergence}{\mathrm{div}\,}  %ダイバージェンス
\newcommand{\grad}{\mathrm{grad}\,}  %グラディエント
\newcommand{\rot}{\mathrm{rot}\,}  %ローテーション

\numberwithin{equation}{section}
\setcounter{equation}{16}

%
% \pagestyle{myheadings}
\begin{document}
%
%
1 AEI 3番 : 馬場 悠斗
\setcounter{section}{12}
\subsection{線形負荷のみの補償に関する方程式}

線形負荷と非線形負荷の並列組み合わせ
図12.1の非線形荷重を取り除いたとする. 
すると, 図12.2では
\begin{equation}
  i_l = 0
\end{equation}

\begin{equation}
  i = i_b = \frac{e_{in}}{\left(Z_a + Z_d\right)}
\end{equation}

添え字'1'は, 基本周波数または電源周波数に対するインピーダンスを示す. 
は, 電流の基本波または電源周波数の高調波に対するインピーダンスを示す. 
合成入力インピーダンスを$Z_a1d1$とする. 
\begin{equation}
  Z_{a1d1} = Z_{a1} + Z_{d1} = |Z_{a1d1}| \angle \Phi_{a1d1}
\end{equation}

ここで
\begin{equation}
  |Z_{a1d1}| = \sqrt{|Z_{a1}|^2 + |Z_{d1}|^2 + 2|Z_{a1}||Z_{d1}| \cos \left(\Phi_{a1} - \Phi_{d1}\right) }
\end{equation}
である
\begin{equation}
  \Phi_{a1d1} = \tan^{-1}\left( \frac{\left| Z_{a1} \right| \sin \Phi_{a1} + \left| Z_{d1} \right| \sin \Phi_{d1} }{\left| Z_{a1} \right| \cos \Phi_{a1} + \left| Z_{d1} \right| \cos \Phi_{d1}}\right)
\end{equation}

ここで, コンデンサ$Z_c$(図12.2)を取り除くとします. 
この条件では
\begin{equation}
  Z_{d1} = Z_{b1}
\end{equation}
(12.8)を(12.22)に組み合わせると
\begin{align}
  i & = \frac{e_{in}}{\left| Z_{a1d1}\angle Z_{a1d1} \right|} = i_b \notag                     \\
    & = \sqrt{2} \frac{E_{in}}{\left|Z_{a1d1}\right|} \sin \left(\omega t - \Phi_{a1d1}\right)
\end{align}

\setcounter{figure}{1}
\begin{figure}[b]
  \begin{center}
    \includegraphics[width=100mm]{./img/circuit.jpg}
    \caption{図12.1において線形部分の等価回路}
    \label{circuit}
  \end{center}
\end{figure}

\newpage

線形, 非補償負荷$Z_b$に対する条件では, 負荷電圧$e_0$は次のように修正されます. 

\begin{equation}
  e_0 = i_b \left(\left|Z_{b1} \angle \Phi_{b1}\right|\right)
\end{equation}

(12.23)と(12.24)を組み合わせると

\begin{equation}
  e_0 = \sqrt{2} \frac{E_{in}}{\left|Z_{a1d1}\right|} \left|Z_{b1}\right| \sin \left(\omega t - \Phi_{a1b1} + \Phi_{b1}\right)
\end{equation}

負荷における電流$i$ は, 純粋な抵抗と純粋なリアクタンスの等価並列負荷が使用された場合に流れる仮想成分$i_R$,$i_x$($e_0$に関して)に分解できる. 

\begin{equation}
  i_R = \sqrt{2} \frac{E_{in}}{\left|Z_{a1b1}\right|} \cos \Phi_{b1} \sin \left(\omega t - \phi_{a1b1} + \Phi_{b1}\right)
\end{equation}

\begin{equation}
  i_X = \sqrt{2} \frac{E_{in}}{\left|Z_{a1b1}\right|} \sin \Phi_{b1} \sin \left(\omega t - \Phi_{a1b1} + \Phi_{b1}\right)
\end{equation}

(12.26)および(12.27)の実効形式を(12.25)の実効形式と組み合わせると, 負荷端子における皮相電力の成分が得られる. 

\begin{equation}
  S_R = \frac{{E_{in}}^2 \left|Z_{b1}\right|}{{\left|Z_{a1b1}\right|}^2}\cos\Phi_{b1}
\end{equation}

\begin{equation}
  S_X = \frac{{E_{in}}^2 \left|Z_{b1}\right|}{{\left|Z_{a1b1}\right|}^2}\sin\Phi_{b1}
\end{equation}

負荷での力率一定値動作では, コンデンサC, 
は, (12.27)の$i_x$を補償し, それによって(12.29)の$S_x$に等しい定格電圧を持たなければならない. 

\begin{equation}
  \left|Z_C\right| = \frac{\left|Z_{b1}\right|}{\sin \Phi_{b1}}
\end{equation}

\begin{equation}
  C = \frac{\sin \Phi_{b1}}{\omega \left|Z_{b1}\right|}
\end{equation}

補償された負荷の端子を見たインピーダンス$Z_{d1}$ は, 次式で与えられる. 

\begin{equation}
  Z_{d1} = \frac{\left|Z_{b1}\right|}{\cos \Phi_{b1}} \angle 0
\end{equation}

完全に補償された負荷の場合, 入力インピーダンス(12.32)は純粋な抵抗性である. 


\newpage
words
  \begin{table}[h]
    \begin{tabular}{lll}
    単語            & 品詞    & 日本語訳                 \\ \hline
    subscript     & 形容詞   & 下に書いた; 下付きの              \\ 
                  & 名詞    & 下付き文字                     \\ \hline
    fundamental    & 形容詞   & 基本の,基礎の,根源の,根本的な. \\ \hline
    uncompensated  & 形容詞   & 補償差入れていない  \\ \hline
    hypothetical    & 形容詞  & 仮説の,仮定の   \\ \hline


    \end{tabular}
    \end{table}

\setcounter{section}{1}
\setcounter{figure}{0}
\section*{式の導出}
\subsection*{式(12.20)の導出}
\begin{figure}[b]
  \begin{center}
    \includegraphics[width=100mm]{./img/vector_Z.jpg}
    \caption{ベクトルのイメージ図}
    \label{vector_Z}
  \end{center}
\end{figure}


各インピーダンスの成分をそれぞれ
$Z_{a1} = \left(Z_{a1x} , Z_{a1y}\right) , Z_{d1} = \left(Z_{d1x} , Z_{d1y}\right) $
とすると, 合成インピーダンス$Z_{a1d1}$の成分は$Z_{a1d1} = \left(Z_x , Z_y\right)$は, 
$Z_x = Z_{a1x} + Z_{d1x} , Z_y = Z_{a1y} + Z_{d1y}$となる. これらの関係を図\ref{vector_Z}に示す. \\
$Z_{a1d1}$の大きさは,\\
\begin{align*}
  \left|Z_{a1d1}\right| & = \sqrt{ {\left(Z_x\right)}^2 + {\left(Z_y\right)}^2 }                                                                                            \\
                        & = \sqrt{ {\left(Z_{a1x} + Z_{d1x}\right)}^2 + {\left(Z_{a1x} + Z_{d1x}\right)}^2 }                                                                \\
                        & = \sqrt{ \left( {Z_{a1x}}^2 + 2Z_{a1x}Z_{d1x} + {Z_{d1x}}^2 \right) + \left( {Z_{a1y}}^2 + 2Z_{a1y}Z_{d1y} + {Z_{d1y}}^2 \right)}                 \\
                        & =  \sqrt{ \left( {Z_{a1x}}^2 + {Z_{a1y}}^2 + {Z_{d1x}}^2  + {Z_{d1y}}^2 \right) + \left(2Z_{a1x}Z_{d1x} + 2Z_{a1y}Z_{d1y}  \right)}               \\
                        & =  \sqrt{ \left( {Z_{a1x}}^2 + {Z_{a1y}}^2 \right) + \left({Z_{d1x}}^2  + {Z_{d1y}}^2 \right) + \left(2Z_{a1x}Z_{d1x} + 2Z_{a1y}Z_{d1y}  \right)} \\
                        & =  \sqrt{ \left( {Z_{a1x}}^2 + {Z_{a1y}}^2 \right) + \left({Z_{d1x}}^2  + {Z_{d1y}}^2 \right) + 2\left(Z_{a1x}Z_{d1x} + Z_{a1y}Z_{d1y}  \right)}  \\
\end{align*}
$\left|Z_{a1}\right| = \sqrt{{{Z_{a1x}}^2 + {Z_{a1y}}^2}},\left|Z_{d1}\right| = \sqrt{{{Z_{d1x}}^2 + {Z_{d1y}}^2}}$である. \\
また, 各インピーダンスをベクトルと考え, この2つのベクトルの成す角を$\Phi_{a1} - \Phi_{d1}$とすると, \\
$\left(Z_a , Z_d\right) = Z_{a1x}Z_{d1x} + Z_{a1y}Z_{d1y} = \left|Z_{a1}\right| \left|Z_{d1}\right| \cos \left(\Phi_{a1} - \Phi_{d1}\right)$であるから, 
これらを代入すると, 
\begin{equation*}
  \left|Z_{a1d1}\right| = \sqrt{ {\left|Z_{a1}\right|}^2 + {\left|Z_{d1}\right|}^2 + 2 \left|Z_{a1}\right| \left|Z_{d1}\right| \cos \left(\Phi_{a1} - \Phi_{d1}\right)}
\end{equation*}
となり, (12.20)が得られる. 

\subsection*{式(12.21)の導出}
まず, tanの定義より, $\tan\left(\Phi_{a1} - \Phi_{d1}\right) = \frac{Z_y}{Z_x}$であるから, 
逆三角関数を用いて, 
$\left(\Phi_{a1} - \Phi_{d1}\right) = \tan^{-1}\frac{Z_y}{Z_x}$.\\
$Z_{a1d1}$の各成分は
$Z_x = Z_{a1x} + Z_{d1x}$ , $Z_y = Z_{a1y} + Z_{d1y}$ .\\
$Z_{a1}$,$Z_{d1}$のx,y成分はそれぞれ\\
$Z_{a1x} = \left|Z_a\right| \cos \Phi_{a1}$ ,$Z_{a1y} = \left|Z_a\right| \sin \Phi_{a1}$\\
$Z_{d1x} = \left|Z_d\right| \cos \Phi_{d1}$ ,$Z_{d1y} = \left|Z_d\right| \sin \Phi_{d1}$\\
であるから, 
\begin{align*}
  \Phi_{a1d1} = \tan^{-1}\left( \frac{\left| Z_{a1} \right| \sin \Phi_{a1} + \left| Z_{d1} \right| \sin \Phi_{d1} }{\left| Z_{a1} \right| \cos \Phi_{a1} + \left| Z_{d1} \right| \cos \Phi_{d1}}\right)
\end{align*}
が得られる. 

\subsection*{式(12.23)の導出}
% まず, 式(12.8)は, 
% \begin{equation*}
%   i_{bx} = \sqrt{2} \sum_{1}^{n} \frac{E_n}{\left|Z_{dn}\right|} \sin \Phi_{dn} \cos \left(n\omega t + \alpha_n\right)
% \end{equation*}
まず, 式(12.18)は, 
\begin{equation*}
  i = i_b = \frac{e_{in}}{\left(Z_a + Z_d\right)}
\end{equation*}
図12.2からキャパシタを除去した場合, $Z_d$部分は,$Z_b$のみとなり, 
並列部分がなくなるため$i = i_b$が成立する. 
これを式(12.18)に代入すると, 電流$i$は, 
$e_{in} = \sqrt{2}E_{in}\sin \omega t$
であるから, オームの法則を用いて, 
\begin{align*}
  i & = i_b = \frac{e_{in}}{Z_{a1d1}} = \frac{e_{in}}{Z_{a1b1}} = \frac{\sqrt{2}E_{in}\sin \left(\omega t\right)}{\left|Z_{a1b1}\right| \angle \Phi_{a1b1}}
  = \frac{ \sqrt{2}{E_{in}} \angle \left( \omega t \right)}{\left| Z_{a1b1} \right| \angle \Phi_{a1b1}} = \sqrt{2} {\left(\frac{E_{in}}{\left|Z_{a1b1}\right|}\right)} \angle{\left(\omega t - \Phi_{a1b1}\right)} \\
    & = \sqrt{2} \frac{ E_{in} }{\left| Z_{a1b1} \right|}\sin \left(\omega t - \Phi_{a1b1}\right) \\
    \because  z_1 &= \left| z_1 \right| \angle \theta_1 = \left| z_1 \right| \left(\cos \theta_1 + j \sin \theta_1\right) \\
              z_2 &= \left| z_2 \right| \angle \theta_2 = \left| z_2\right| \left(\cos \theta_2 + j \sin \theta_2\right)  \\
              z_1z_2 &= \left| z_1 \right|\left| z_2 \right| \angle \left(\theta_1 + \theta_2\right) = \left| z_1 \right|\left| z_2\right| \left(\cos \left(\theta_1 + \theta_2\right) + j \sin \left(\theta_1 + \theta_2\right)\right)
\end{align*}

が得られる. 

\subsection*{式(12.25)の導出}
式(12.23)及び, (12.24)から, 
\begin{equation*}
  \left\{ \,
  \begin{aligned}
     & i = i_b = \sqrt{2} \frac{ E_{in} }{\left| Z_{a1b1} \right|}\sin \left(\omega t - \Phi_{a1b1}\right) & \left(12.23\right) \\
     & e_0 = i_b \left(\left|Z_{b1} \angle \Phi_{b1}\right|\right)                                         & \left(12.24\right)
  \end{aligned}
  \right.
\end{equation*}
であるから, 
\begin{align*}
  e_0 & = \sqrt{2} \frac{ E_{in} }{\left| Z_{a1b1} \right|}\sin \left(\omega t - \Phi_{a1b1}\right) \cdot \left(\left|Z_{b1}\right| \angle \Phi_{b1}\right) 
  = \sqrt{2} \frac{ E_{in} }{\left| Z_{a1b1} \right|} \left|Z_{b1}\right| \sin {\left(\omega t - \Phi_{a1b1} + \Phi_{b1}\right)}
\end{align*}
が得られる. 

\subsection*{式(12.26),式(12.27)の導出}
式(12.23)から
\begin{align*}
  i = i_b = \sqrt{2} \frac{ E_{in} }{\left| Z_{a1b1} \right|}\sin \left(\omega t - \Phi_{a1b1}\right) & \left(12.23\right) \\
\end{align*}
$Z_b$における電流の抵抗成分とリアクタンス成分に分解するためには、$\cos \Phi_{b1}$成分と$\sin\Phi_{b1}$成分に分解すればよいので、
\begin{equation*}
  \left\{ \,
  \begin{aligned}
  i_R &= i_b \cdot \cos \Phi_{b1} = \sqrt{2} \frac{ E_{in} }{\left| Z_{a1b1} \right|} \cos \Phi_{b1} \sin \left(\omega t - \Phi_{a1b1}\right)\\
  i_X &= i_b \cdot \sin \Phi_{b1} = \sqrt{2} \frac{ E_{in} }{\left| Z_{a1b1} \right|} \sin \Phi_{b1} \sin \left(\omega t - \Phi_{a1b1}\right)
  \end{aligned}  
  \right .
\end{equation*}
が得られる。
\[
  \left(
    \begin{tabular}{l}
      $e_0 = \sqrt{2} \frac{ E_{in} }{\left| Z_{a1b1} \right|} \left|Z_{b1}\right| \sin {\left(\omega t - \Phi_{a1b1} + \Phi_{b1}\right)}$\\ \\
      から、$\left|Z_{b1}\right| \angle \Phi_{b1}$を用いて\\ \\
      $i_R = \frac{e_0}{\left|Z_{b1}\right| \angle \Phi_{b1}} \cdot \cos \Phi_{b1} = \sqrt{2} \frac{ E_{in} }{\left| Z_{a1b1} \right|} \cos \Phi_{b1} \sin {\left(\omega t - \Phi_{a1b1}\right)}$\\ \\
      $i_X = \frac{e_0}{\left|Z_{b1}\right| \angle \Phi_{b1}} \cdot \sin \Phi_{b1} = \sqrt{2} \frac{ E_{in} }{\left| Z_{a1b1} \right|} \sin \Phi_{b1} \sin {\left(\omega t - \Phi_{a1b1}\right)}$\\ \\
      導出しても同様になる。
    \end{tabular}
  \right)
\]


\subsection*{式(12.28),式(12.29)の導出}
皮相電力は、電流と電圧の積であるから、抵抗とリアクタンスのそれぞれの皮相電力は、
\begin{align*}
  ei_R  &= e_0 i_R = \sqrt{2} \frac{ E_{in} }{\left| Z_{a1b1} \right|} \left|Z_{b1}\right|  \sin \left(\omega t - \Phi_{a1b1}\right) \cdot \sqrt{2} \frac{ E_{in} }{\left| Z_{a1b1} \right|} \cos \Phi_{b1} \sin \left(\omega t - \Phi_{a1b1}\right) \\
        &= 2 \frac{{E_{in}}^2}{{\left|Z_{a1b1}\right|}^2}\left|Z_{b1}\right| \cos \Phi_{b1} {\left(\sin \left( \omega t - \Phi_{a1b1}\right)\right)}^2 \\
  ei_X  &= e_0 i_X = \sqrt{2} \frac{ E_{in} }{\left| Z_{a1b1} \right|} \left|Z_{b1}\right| \sin \left(\omega t - \Phi_{a1b1}\right) \cdot \sqrt{2} \frac{ E_{in} }{\left| Z_{a1b1} \right|} \sin \Phi_{b1} \sin \left(\omega t - \Phi_{a1b1}\right) \\
        &= 2 \frac{{E_{in}}^2}{{\left|Z_{a1b1}\right|}^2} \left|Z_{b1}\right| \sin \Phi_{b1} {\left(\sin \left( \omega t - \Phi_{a1b1}\right)\right)}^2 \\
\end{align*}
本文中から、実行形式(rms form)とあるので、実効値の定義から2乗平均の平方根を考える。\\
電圧$e_0$、電流$i_b$に関しては、式(12.1)、式(12.4)からすでに実効値である。
$\cos \Phi_{b1}$及び、$\sin \Phi_{b1}$に関しては定数であるため実効値を考慮しなくて良い。
つまり、${\left(\sin \left( \omega t - \Phi_{a1b1}\right)\right)}^2$の項の実効値を考えればよい。
$\left(\sin \left( \omega t - \Phi_{a1b1}\right)\right)$の実効値は、
\begin{align*}
    & \sqrt{\frac{1}{T} \int_{0}^{T}  \sin^2 \left( \omega t - \Phi_{a1b1}\right)} dt               &\because \text{(半角の公式)}\\
  = & \sqrt{ \frac{1}{2T} \int_{0}^{T}  1 - \cos 2 \left( \omega t - \Phi_{a1b1}\right)} dt         \\
  = & \sqrt{ \frac{1}{2T} \left[  1 - \sin 2 \left( \omega t - \Phi_{a1b1}\right) \right]_{0}^{T}}  \\
  = & \sqrt{ \frac{1}{2T} \left\{ \left(T - \frac{1}{2\omega} \sin 2 \left( \omega T - \Phi_{a1b1}\right)\right) - 
  \left(0 - \frac{1}{2\omega} \sin 2 \left( 0 - \Phi_{a1b1}\right)\right) \right\} }  
  &\because  \sin \left(\omega t - \Phi_{a1b1}\right)\text{の周期は}\frac{2\pi}{\omega}\\
  = & \sqrt{ \frac{1}{2T} \left\{ \left(T - \frac{1}{2\omega} \sin \left(2 \omega \frac{2\pi}{\omega} - 2\Phi_{a1b1}\right)\right) + 
  \frac{1}{2\omega} \sin \left(-2\Phi_{a1b1}\right) \right\} }  \\
  = & \sqrt{ \frac{1}{2T} \left\{ \left(T - \frac{1}{2\omega} \sin \left(4\pi - 2\Phi_{a1b1}\right)\right) + 
  \frac{1}{2\omega} \sin \left(-2\Phi_{a1b1}\right) \right\} }  &\because \sin \left(2n\pi + \theta\right) = \sin \left(\theta\right)\\
  = & \sqrt{ \frac{1}{2T} \left\{ \left(T - \frac{1}{2\omega} \sin \left(- 2\Phi_{a1b1}\right)\right) + 
  \frac{1}{2\omega} \sin \left(-2\Phi_{a1b1}\right) \right\} }  = \frac{1}{\sqrt{2}} \\
\end{align*}
となる。式の$\left(\sin \left( \omega t - \Phi_{a1b1}\right)\right)$を$\frac{1}{\sqrt{2}}$に置き換えると、

\begin{align*}
  ei_R  &= e_0 i_R = 2 \frac{{E_{in}}^2}{{\left|Z_{a1b1}\right|}^2} \left|Z_{b1}\right| \cos \Phi_{b1} {\left(\sin \left( \omega t - \Phi_{a1b1}\right)\right)}^2 \\
        &= 2 \frac{{E_{in}}^2}{{\left|Z_{a1b1}\right|}^2} \left|Z_{b1}\right| \cos \Phi_{b1} \left({\frac{1}{\sqrt{2}}}\right)^2 = \frac{{E_{in}}^2 \left|Z_{b1}\right|}{{\left|Z_{a1b1}\right|}^2}\cos\Phi_{b1} = S_R  \\
  ei_X  &= e_0 i_X = 2 \frac{{E_{in}}^2}{{\left|Z_{a1b1}\right|}^2} \left|Z_{b1}\right| \sin \Phi_{b1} {\left(\sin \left( \omega t - \Phi_{a1b1}\right)\right)}^2 \\
        &= 2 \frac{{E_{in}}^2}{{\left|Z_{a1b1}\right|}^2} \left|Z_{b1}\right| \sin \Phi_{b1} \left({\frac{1}{\sqrt{2}}}\right)^2 = \frac{{E_{in}}^2 \left|Z_{b1}\right|}{{\left|Z_{a1b1}\right|}^2}\sin\Phi_{b1} = S_X  \\
\end{align*}
が得られる。

\subsection*{式(12.30),式(12.31)の導出}
キャパシタの並列接続によって、インダクタンスによる位相のズレを無くしたい。
インダクタンスによる位相のずれは$\frac{\pi}{2}$であるのに対し、
キャパシタによる位相のずれは$-\frac{\pi}{2}$であるため、
それぞれの素子による電力の大きさが等しければ打ち消すことができる。
ここで、キャパシタはインダクタンスに対して並列接続に接続するため、
それぞれの素子にかかる電圧は等しくなる。
つまり、それぞれのインピーダンスの大きさが等しければ打ち消すことができるということになる。
キャパシタによる電力$Q_C$は、キャパシタにかかる電圧を$V_c$とすると、
\begin{align*}
  Q_C &= \frac{{V_c}^2}{\left|Z_c\right|} &\because \text{$V_c = V_b$($V_b$はリアクタンスにかかる電圧)} \\
  Q_C &= \frac{{V_b}^2}{\left|Z_c\right|}
\end{align*}
これが、リアクタンスによる皮相電力$S_X$と等しければよいので、
\begin{align*}
  Q_C &= \frac{{V_b}^2}{\left|Z_c\right|} = \frac{{E_{in}}^2 \left|Z_{b1}\right|}{{\left|Z_{a1b1}\right|}^2}\sin\Phi_{b1} 
  &\because \text{分圧則} V_b = E_{in} \cdot \frac{\left|Z_{b1}\right|}{\left|Z_{a1b1}\right|}\\
  \frac{{V_b}^2}{\left|Z_c\right|} &= \frac{{E_{in}}^2 \left|Z_{b1}\right|}{{\left|Z_{a1b1}\right|}^2}\sin\Phi_{b1}\\
  \frac{{\left(E_{in} \cdot \frac{\left|Z_{b1}\right|}{\left|Z_{a1b1}\right|}\right)}^2}{\left|Z_c\right|} &= \frac{{E_{in}}^2 \left|Z_{b1}\right|}{{\left|Z_{a1b1}\right|}^2}\sin\Phi_{b1}\\
  \frac{{E_{in}}^2 {\left|Z_{b1}\right|}^2}{\left|Z_c\right| {\left|Z_a1b1\right|}^2} &= \frac{{E_{in}}^2 \left|Z_{b1}\right|}{{\left|Z_{a1b1}\right|}^2}\sin\Phi_{b1}\\
  \frac{\left|Z_{b1}\right|}{\left|Z_c\right|} &= \sin \Phi_{b1} \\
  \therefore \left|Z_c\right| &= \frac{\left|Z_{b1}\right|}{\sin \Phi_{b1}}\\
\end{align*}
また、
\begin{align*}
  \left|Z_c\right| = \frac{1}{\omega C} = \frac{\left|Z_{b1}\right|}{\sin \Phi_{b1}}\\
  \therefore C &= \frac{ \sin \Phi_{b1}}{\omega \left|Z_[b1]\right|}\\
\end{align*}
が得られる。

\subsection*{式(12.32)の導出}
まず、インピーダンス$Z_{b1} = R + jX$とすると$Z_{b1} = \left|Z_{b1}\right| \left(\cos \Phi_{b1} + j\sin\Phi_{b1}\right)$と表せる。
また、インピーダンス$Z_c$は$-j\left|Z_c\right|$と表せる。
これらから、補償後の合成インピーダンス$Z_d$は、
\begin{align*}
  \frac{1}{Z_d} &= \frac{1}{Z_b1} + \frac{1}{Z_C} 
  = \frac{1}{\left|Z_{b1}\right| \left(\cos \Phi_{b1} + j\sin\Phi_{b1}\right) } + \frac{1}{-j \frac{\left| Z_{b1} \right|}{\sin \Phi_{b1}}}
  &\because \text{式(12.30)}\\
                &= \frac{1}{\left|Z_{b1}\right|} \left(\cos \Phi_{b1} - j\sin\Phi_{b1}\right) + \frac{1}{\left| Z_{b1} \right|}j\sin \Phi_{b1}\\
  \frac{1}{Z_d} &= \frac{1}{\left|Z_{b1}\right|}\cos \Phi_{b1} \\
  \therefore Z_d &=  \frac{\left|Z_{b1}\right|}{\cos \Phi_{b1}} \angle 0
\end{align*}
が得られる。
\end{document}